\documentclass[12pt]{article}
\usepackage{titlesec}
\usepackage{hyperref}
\usepackage{graphicx}
\usepackage{booktabs}
\titlelabel{\thetitle.\quad}
\title{Applied Numerical Computing\\ for Scientists and Engineers}
\date{In-Class Problem: \LaTeX{} Basics}
\author{Dr. Ford Versypt}
\begin{document}

\maketitle

Use the built-in article class in \LaTeX{} and the use the booktabs package as in the tutorial with Overleaf as the editor to create a document that has a table and a cross-reference to a table.

\section{Part 1}
\begin{enumerate}
\item Write the \LaTeX{} code to reproduce Table \ref{table:first}. Note that the parameters use the inline math notation surrounded by pairs of \$'s and that superscripts or subscripts require inline math notation.
\item Write the \LaTeX{} code to cross reference Table \ref{table:first}.
\end{enumerate}


\begin{table}[htbp]
\begin{center}
\caption{Parameter values for two cases}
\begin{tabular}{llll}\toprule
Parameter & Units & Case 1 & Case 2 \\ \midrule
$a$ & h & 1.0 & 1.0 \\
$b_1$ & h & 1.0  & 1.0 \\
$b_2$ & ng h ml$^{-1}$ & 1.0 & 1.0 \\ 
\bottomrule
\end{tabular}
\label{table:first}
\end{center}
\end{table} 

\section{Part 2}
\begin{enumerate}
\item Write the \LaTeX{} code to reproduce Table \ref{table:multicolumn}. You are welcome to search the internet for ideas, packages, or code snippets for formatting multicolumn tables.
\item Write the \LaTeX{} code to cross reference Table \ref{table:multicolumn}.
\end{enumerate}

\begin{table}[htbp]
\begin{center}
\caption{Parameter values for two cases with some values reported and others calculated}
\begin{tabular}{llll}\toprule
Parameter & Units & Case 1 & Case 2 \\ \midrule
\multicolumn{4}{c}{Reported in the literature}\\ \midrule
$a$ & h & 1.0 & 1.0 \\
$b_1$ & h & 1.0  & 1.0 \\
$b_2$ & ng h ml$^{-1}$ & 1.0 & 1.0 \\ \midrule
\multicolumn{4}{c}{Calculated from literature values}\\ \midrule
$c$ & h$^{-1}$ & 2.0 & 2.0  \\
$d$ & h$^{-1}$ & 2.0 & 2.0 \\
\bottomrule
\end{tabular}
\label{table:multicolumn}
\end{center}
\end{table}  

\end{document}